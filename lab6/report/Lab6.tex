\documentclass[10pt,a4paper,oneside]{article}
\usepackage[utf8]{inputenc}
\usepackage{amsmath}
\usepackage{amsfonts}
\usepackage{amssymb}
\usepackage{listings}
\usepackage[russian]{babel}%Подключаем русский язык.
\usepackage{graphicx}
\usepackage{geometry} % Меняем поля страницы.
\geometry{left=3cm} %Левое поле.
\geometry{right=2cm} %Правое поле.
\geometry{top=3cm} %Верхнее поле.
\geometry{bottom=2cm} %Нижнее поле.

%Начало документа
\begin{document}
\begin{titlepage}
	\begin{center}
		Санкт-Петербургский Политехнический Университет     Петра Великого \\
		
		Институт компьютерных наук и технологий \\
		
		Кафедра компьютерных систем и программных технологий
	\end{center}
	
	\vfill
	
	\begin{center}
		Лабораторная работа №6\\
		по теме\\
		"Цифровая модуляция"\\
	\end{center}
	
	\vfill
	
	\newlength{\ML}
	\settowidth{\ML}{«\underline{\hspace{0.7cm}}» \underline{\hspace{2cm}}}
	\hfill\begin{minipage}{0.4\textwidth}
		Выполнил студент группы 33501/3\\
		\underline{\hspace{\ML}} Ромащенко Д.\,Ю\\
	\end{minipage}%
	
	\bigskip
	
	\settowidth{\ML}{«\underline{\hspace{0.7cm}}» \underline{\hspace{2cm}}}
	\hfill\begin{minipage}{0.4\textwidth}
		Руководитель\\
		\underline{\hspace{\ML}} Богач Н.\,В\\
	\end{minipage}%
	
	\vfill
	
	\begin{center}
		Санкт-Петербург\\
		2018 
	\end{center}
	
\end{titlepage}


%Первый параграф - Цель работы.
\section{Цель работы}
Изучение методов модуляции цировых сигналов.

%Второй параграф - Постановка задачи.
\section{Постановка задачи}
\begin{enumerate}
\item Получить сигналы BPSK, PSK, OQPSK, genQAM, MSK модуляторов.
\item Построить их сигнальные созвездия.
\item Провести сравнение изученных методов модуляции цифровых сигналов.
\end{enumerate}

%Третий параграф - Теоретическая часть.
\section{Теоретическая часть}
Цифровая модуляция и демодуляция включают в себя две стадии. При модуляции цифровое сообщение сначала преобразуется в аналоговый модулирующий сигнал с помощью функции modmap, а затем осуществляется аналоговая модуляция. При демодуляции сначала получается аналоговый демодулированный сигнал, а затем он преобразуется в цифровое сообщение с помощью функции demodmap.

Аналоговый несущий сигнал модулируется цифровым битовым потоком.
Существуют три фундаментальных типа цифровой модуляции (или шифтинга) и один гибридный:
\begin{enumerate}
	\item ASK – Amplitude shift keying (Амплитудная двоичная модуляция).
	\item FSK – Frequency shift keying (Частотая двоичная модуляция).
	\item PSK – Phase shift keying (Фазовая двоичная модуляция).
	\item ASK/PSK.
\end{enumerate}
Одна из частных реализаций схемы ASK/PSK, которая называется QAM - Quadrature Amplitude Modulation (квадратурная амплитудная модуляция (КАМ). Это метод объединения двух AM-сигналов в одном канале. Он позваляет удвоить эффективную пропускную способность. В QAM используется две несущих с одинаковой частотой но с разницей в фазе на четверть периода (отсюда и возникает слово квадратура). 
Частотная модуляция представляет логическую единицу интервалом с большей частотой, чем ноль.
Фазовый шифтинг представляет «0» как сигнал без сдвига, а «1» как сигнал со сдвигом.

BPSK : используется единственный сдвиг фазы между «0» и «1» — 180 градусов, половина периода.
Существуют также QPSK:
QPSK использует 4 различных сдвига фазы (по четверти периода) и может кодировать 2 бита в символе (01, 11, 00, 10). 
\newpage
\section{Ход работы}
Реализуем различные типы модуляции в Matlab:
\begin{verbatim}
%BPSK
h = modem.pskmod('M', 4);
g = modem.pskdemod('M', 4);
msg = randi([0 3],10,1)
modSignal = modulate(h,msg);
errSignal = (randerr(1, 10, 3) ./ 30)';
modSignal = modSignal + errSignal;
demodSignal = demodulate(g,modSignal);
scatterplot(modSignal)
\end{verbatim}

\begin{figure}[h]
	\center{\includegraphics[width=0.6\linewidth]{BPSK}} 
	\caption{Сигнальное созвездие BPSK}
\end{figure}
\newpage
\begin{verbatim}
%PSK
h = modem.pskmod('M', 8);
g = modem.pskdemod('M', 8);
msg = randi([0 7],10,1)
modSignal = modulate(h,msg);
errSignal = (randerr(1, 10, 3) ./ 30)';
modSignal = modSignal + errSignal;
demodSignal = demodulate(g,modSignal);
scatterplot(modSignal)
\end{verbatim}

\begin{figure}[h]
	\center{\includegraphics[width=0.6\linewidth]{PSK}} 
	\caption{Сигнальное созвездие PSK}
\end{figure}

\newpage
\begin{verbatim}
%OQPSK
h = modem.oqpskmod;
g = modem.oqpskdemod;
msg = randi([0 99],1,4)
modSignal = modulate(h,msg);
errSignal = (randerr(1, 200, 100) ./ 30)';
modSignal = modSignal + errSignal;
demodSignal = demodulate(g,modSignal);
scatterplot(modSignal)
\end{verbatim}
\begin{figure}[h]
	\center{\includegraphics[width=0.6\linewidth]{OQPSK}} 
	\caption{Сигнальное созвездие OQPSK}
\end{figure}

\newpage
\begin{verbatim}
%genQAM
M = 10;
h = modem.genqammod('Constellation', exp(j*2*pi*[0:M-1]/M));
g = modem.genqamdemod('Constellation', exp(j*2*pi*[0:M-1]/M));
msg = randi([0 9],100,1);
modSignal = modulate(h,msg);
errSignal = (randerr(1,100, 3) ./ 30)';
modSignal = modSignal + errSignal;
demodSignal = demodulate(g,modSignal);
scatterplot(modSignal);
\end{verbatim}
\begin{figure}[h]
	\center{\includegraphics[width=0.6\linewidth]{genQAM}} 
	\caption{Сигнальное созвездие genQAM}
\end{figure}

\newpage
\begin{verbatim}
%MSK
h = modem.mskmod('SamplesPerSymbol', 10);
g = modem.mskdemod('SamplesPerSymbol', 10);
msg = randi([0 1],10,1);
modSignal = modulate(h, msg);
errSignal = (randerr(1, 100, 3) ./ 30)';
modSignal = modSignal + errSignal;
demodSignal = demodulate(g, modSignal);
scatterplot(modSignal);
\end{verbatim}
\begin{figure}[h]
	\center{\includegraphics[width=0.6\linewidth]{MASK}} 
	\caption{Сигнальное созвездие MSK}
\end{figure}


\section{Вывод\\\\}
В данной работе мы изучили различные виды фазовой, частотной и амплитудной модуляции цифровых сигналов.
В фазовой модуляции PSK фаза несущего колебания имеет фиксированные позиции с одинаковым шагом, соответствующие цифровой посылке. С этим видом модуляции возникают проблемы синхронизации из-за трудности однозначной интерпретации поворота созвездия.
Квадратурная амплитудная модуляция характеризуется изменением и фазы, и амплитуды, из-за чего растёт информационная плотность.
Модуляция с минимальным сдвигом является примером частотной манипуляции, в ней нет фазовых ступеней, а частота изменяется в момент пересечения несущей нулевого уровня. Бинарный PSK хоть и имеет низкую плотность информации ввиду всего двух ступеней фаз, но обладает большой помехоустойчивостью благодаря большой дистанции между этими двумя состояниями.
\end{document}