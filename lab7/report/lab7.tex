\documentclass[10pt,a4paper,oneside]{article}
\usepackage[utf8]{inputenc}
\usepackage{amsmath}
\usepackage{amsfonts}
\usepackage{amssymb}
\usepackage{graphicx}
\usepackage[russian]{babel}% русский язык.
\usepackage{geometry} % Меняем поля страницы.
\geometry{left=3cm} %Левое поле.
\geometry{right=2cm} %Правое поле.
\geometry{top=3cm} %Верхнее поле.
\geometry{bottom=2cm} %Нижнее поле.

\begin{document}

\begin{titlepage}
	\newpage
	%Название ВУЗа и институт.
	\begin{center}
		\Large Санкт-Петербургский Государственный Политехнический Университет\\
		Институт Компьютерных Наук и Программных Технологий\\
	\end{center}
	%Кафедра.
	\begin{center}
		\large\textbf {Кафедра Компьютерных Систем и Програмных Технологий}
	\end{center}
	
	%Пропуск места. 
	\vspace{10em}
	%!!!!!!!!!!!!!!!!!!!!!!!!!!!!!!!!!Название работы.
	\begin{center}
		\large{Отчёт по лабораторной работе №7 \\ на тему \\
			\textbf{Помехоустойчивые коды} }
	\end{center}
	
	%Делаем пропуск и пишем студента и преподавателя.
	\vspace{20em}
	\begin{flushright}
		\textbf{Работу выполнил\\}Студена группы 33501/3 \\ Ромащенко Д.Ю.\\
		\textbf{Преподаватель\\}Богач Н.В. 
	\end{flushright}
	
	\vspace{\fill}%В самом низу
	\begin{center}
		Санкт-Петербург \\ 2018
	\end{center}
\end{titlepage} %Закончили титульный лист.


\section{Постановка задачи}

\hspace{0,5cm} Провести кодирование/декодирование сигнала кодом Хэмминга 2мя способами с помощью встроенных функций encode/decode, а также через создание проверочной и генераторной матриц и вычисление синдрома. Провести кодирование/декодирование с помощью циклических кодов

\section{Теоретическое обоснование}


\hspace{0,5cm}  Кодирование передаваемого сообщения позволяет осуществлять его проверку на наличие ошибок при получении, а в некоторых случаях и исправлять их. Данная возможность достигается за счет введения информационной избыточности, что уменьшает удельное количество полезной информации в сообщении.\\


Значительную долю кодов составляют \textbf{блочные коды}. При их применении передаваемое сообщение разбивается на блоки одинаковой длины, после чего каждому блоку сопоставляется код в соответствии с выбранным способом кодирования.\\


Другая характеристика, позволяющая выделить коды в отдельный класс - цикличность. У кодов этого класса циклическая перестановка букв слова также является кодовым словом.\\


\textbf{Код Хэмминга} является циклическим самокорректирующимся кодом.
Помимо информационных бит в сообщении передается набор контрольных бит, которые вычисляются как сумма по модулю 2 всех информационных бит, кроме одного. Для $m$ контрольных бит максимальное число информационных бит составляет $2^m-n-1$. Код Хэмминга позволяет обнаружить до двух ошибок при передаче и исправить инверсную передачу одного двоичного разряда. \\


\textbf{Коды БЧХ} позволяют при необходимости исправлять большее число ошибок в разрядах за счет внесения дополнительной избыточности. Они принадлежат к  категории блочных кодов. Частным случаем БЧХ кодов являются коды \textbf{Рида-Соломона}, которые работают с недвоичными данными. Их корректирующая способность, соответственно, не ниже, чем у кодов Хэмминга.\\


Кодировку сообщения производят с помощью генераторной матрицы, домножение на которую столбца создает кодированное сообщение. На приемной стороне сообщение домножается на проверочную матрицу, полученный результат называется синдромом и позволяет определить наличие ошибок и их местоположение, если корректирующая способность кода достаточна.\\


\section{Ход работы}


\subsection{Код Хэмминга}

Произведем кодирование/декодирование сигнала кодом Хэмминга с
помощью встроенных функций encode/decode.

 \begin{verbatim}
message = [0 1 1 0]

code = encode(message,7,4)
code(3) = not(code(3))
[dec,err] = decode(code,7,4)
 \end{verbatim}

При передаче сообщения без ошибки количество ошибок равно нулю.
При допущении ошибки в 3 бите обнаруживается одна ошибка err = 1, которая исправляется и сообщение декодируется верно.
 \begin{verbatim}
code =

     1     0     0     0     1     1     0


code =

     1     0     1     0     1     1     0
 \end{verbatim}
Произведем кодирование/декодирование сигнала кодом Хэмминга через
создание проверочной и генераторной матриц и вычисление
синдрома.

 \begin{verbatim}
message = [0 1 1 0]
[h,g,n,k] = hammgen(3)

m = message*g
m = rem(m,ones(1,n).*2)

m(3) = not(m1(3))
synd = m*h'
synd = rem(synd,ones(1,n-k).*2)

stbl = syndtable(h)
tmp = bi2de(synd,'left-msb')
z = stbl(tmp+1,:)
rez = xor(m,z)
 \end{verbatim}

Синдром был вычислен домножением на матрицу h', после чего с помощью матрицы синдрома выявляем ошибочный бит в посылке и исправляем его:
 \begin{verbatim}
mes = 1     0     \textbf{1}     0     1     1     0

z   = 0     0     \textbf{1}     0     0     0     0

rez = 1     0     \textbf{0}    0     1     1     0
 \end{verbatim}
Исправляющая способность кода равна 1.

\subsection{Циклический код}

Выполним кодирование и декодирование сообщения [0 1 1 0]:

 \begin{verbatim}
message = [0 1 1 0]
pol = cyclpoly(7,4)
[h,g] = cyclgen(7,pol)

code = message*g;
code = rem(code,ones(1,n).*2)

code(4) = not(code(4))


synd = code*h'
synd = rem(synd,ones(1,n-k).*2)

stbl = syndtable(h)
tmp = bi2de(synd,'left-msb')
z = stbl(tmp+1,:)
rez = xor(code,z)

 \end{verbatim}
Сначала строится порождающий полином циклического
кода: $x^3+x+1$.  Далее, использовав этот полином в качестве одного из параметров
функции cyclgen, получили порождающую и проверочную матрицы для данного кода. 
В результате сообщение было закодировано следующим образом: [0   0   1   0   1   1   0].
Допущенная ошибка в 4 разряде была успешно обнаружена и исправлена синдромом.

Исправляющая способность кода равна 1.


\subsection{Коды БЧХ}

Произведем кодирование и декодирование сообщения [0 0 1 0] при помощи кодов БЧХ:

 \begin{verbatim}
msg = [0 1 1 0]

codebch = comm.BCHEncoder(7,4)
decbch = comm.BCHDecoder(7,4)
temp = message';
code = step (codebch , temp(:))'

code(4) = not(code(4))
decode = step (decbch , code')'

 \end{verbatim}
Закодированное сообщение: [0   1   1   1   0   0  1]
Допущенная ошибка в 4 разряде была успешно успешно обнаружена и исправлена.

Исправляющая способность кода равна 1.

При  k = 7, n = 15 корректирующая способность кода БЧХ стала равна 2, что позволило исправлять 2 ошибки.

\subsection{Коды Рида-Соломона}

Произведем кодирование и декодирование посылки при помощи кодов Рида-Соломона. Количество информационных бит равно 3, количество бит на символ 3, общее число бит таким образом будет равно 7.

 \begin{verbatim}
l = 3; 
n = 7; 
k = 3; 
m = 3; 

msg = gf(randi([0 2m-1],l,k),m)
code = rsenc(msg,n,k)
errs = gf([0 0 0 4 0 0 0; 2 0 0 0 2 0 0; 3 4 5 0 0 0 0 ],m);
code = code + errs

[dec,errnum] = rsdec(code,n,k)
 \end{verbatim}

В первом слове была допущена одна ошибка, во втором и третьем по две.
Исходные посылки:
 \begin{verbatim}

[ 3           7           5]
[2            1           3]
[ 6           1           7]
 \end{verbatim}
Закодированные посылки:
 \begin{verbatim}
[ 3           7           5           0           1            6          0]
[ 0           1           3           6           2           4           5]
[ 5           5           2           4           0           2           3]
 \end{verbatim}
 
 В результате все ошибки были успешно обнаружены и исправлены. Корректирующая способность кода равна 2.

\section{Вывод}
В ходе данной работы были полученны навыки кодирования цифровых сигналов. Кодирование таких сигналов происходить по принципу избыточности. В зависимости от различных задач следует использовать разные коды. Код Хэмминга просто и быстрореализуем, но он предлагает исправления одной ошибки, когда как код Рида-Соломона способен оперировать десятичными числами и способен параллельно обрабатывать несколько потоков.
\end{document}